\documentclass[main.tex]{subfiles}
\begin{document}
\subsection{Day 13: 9/21/22}
\subsubsection{Orthogonal Complements}
\begin{definition}
    Let $V$ be a subspace of $\RR^n$. Then the \vocab{orthogonal complement} $V^\perp$ of $V$ is defined as $\{\vec{x}\in \RR^n : \vec{x}\cdot \vec{v} = 0\forall \vec{v}\in V\}$.
\end{definition}

\begin{proposition}
    $V^\perp$ is a subspace of $\RR^n$.
\end{proposition}
\begin{proof}
    Can just check subspace conditions, but will be proved more directly later on.
\end{proof}

Some examples of $V^\perp$:
\begin{enumerate}
    \item $(\RR^n)^\perp = \{\vec{x} \in \RR^n :  \vec{x}\cdot \vec{v} = 0\forall \vec{v}\in \RR^n\} = \{\vec{0}\}$.
    \item $\{\vec{0}\}^\perp = \{\vec{x} \in \RR^n :  \vec{x}\cdot \vec{0} = 0\} = \RR^n$.
    \item Let $V = \vspan\{\vec{v}_1, \ldots , \vec{v}_m\}$. Then $V^\perp = \{\vec{x} \in \RR^n :  \vec{x}\cdot \vec{v} = 0\forall \vec{v}\in V\}$. Problem Set 5 will show that this set is equal to
    \begin{align*}
    &\{\vec{x} \in \RR^n : \vec{x}\cdot \vec{v}_1 = \vec{x}\cdot \vec{v}_2 = \cdots = \vec{x}\cdot \vec{v}_m = 0\} \\
    &=\left\{\vec{x}\in \RR^n : (\vec{v}_1^\top, \ldots , \vec{v}_m^\top)^\top \vec{x} = (0, \ldots , 0)^\top \right\} \\
    &= N\left(\begin{pmatrix}
        \vec{v}_1^\top \\
        \vdots \\
        \vec{v}_m^\top
    \end{pmatrix}\right).
    \end{align*}Incidentally, this shows that $V^\perp$ is a subspace since null spaces are subspaces.
\end{enumerate}

\begin{theorem}
    Let $A$ be a $m\times n$ matrix. Then $N(A^\top) = (C(A))^\perp$.
\end{theorem}
\begin{proof}
    Denote the columns of $A$ by $\vec{\alpha}_1, \ldots , \vec{\alpha}_n\in \RR^m$. Then
    \begin{align*}
        (C(A))^\perp &= (\vspan\{\vec{\alpha}_1, \ldots , \vec{\alpha}_n\})^\perp \\
        &= N((\vec{\alpha}_1^\top, \ldots, \vec{\alpha}_m^\top)^\top) \\
        &= N(A^\top).\qedhere
    \end{align*}
\end{proof}

\begin{proposition}
    Suppose $W$ is a subspace of $\RR^n$ with $W^\perp = \{\vec{0}\}$. Then $W = \RR^n$.
\end{proposition}

\begin{proof}
    Note that $W$ must be nontrivial because we've already established that $\{0\}^\perp = \RR^n\neq \{\vec{0}\}$; thus, we can pick a basis $\vec{w}_1, \ldots , \vec{w}_k$ for $W$. Then by definition we have $W^\perp = (\vspan\{\vec{w}_1, \ldots , \vec{w}_k\})^\perp = N((\vec{w}_1^\top, \ldots, \vec{w}_k^\top)^\top)$, so 
    \begin{align*}
    &0 = \dim\{\vec{0}\} = \dim W^\perp \\
    &= \dim N((\vec{w}_1^\top, \ldots, \vec{w}_k^\top)^\top) \\
    &= n - \dim C((\vec{w}_1^\top, \ldots, \vec{w}_k^\top)^\top) \\
    &\ge n - k
    \end{align*}
    because $C((\vec{w}_1^\top, \ldots, \vec{w}_k^\top)^\top)$ is a subspace of $\RR^k$, so its dimension is $\le k$ by the Basis Theorem. Thus, we have $k\ge n$. However, $k = \dim W \le n$ because $W$ is a subspace of $\RR^n$, so in fact $k = n\implies \dim W = \dim \RR^n\implies W = \RR^n$ because $W$ is a subspace of $\RR^n$ and Problem Set 3.
\end{proof}

\begin{theorem}
    Suppose $V$ is a subspace of $\RR^n$. Then
    \begin{enumerate}
        \item $V \cap V^\perp = \{\vec{0}\}$.
        \item $\dim V + \dim V^\perp = n$.
        \item $(V^\perp)^\perp = V$.
    \end{enumerate}
\end{theorem}
\begin{proof}
    \begin{enumerate}
        \item Let $\vec{x}\in V\cap V^\perp$. Then $\norm{x}^2 = \vec{x}\cdot \vec{x} = 0$ by definition of $V^\perp$, so $\vec{x} = \vec{0}$, hence $V\cap V^\perp = \{\vec{0}\}$.
        \item For next class.
        \item For next class. \qedhere
    \end{enumerate}
\end{proof}
\end{document}