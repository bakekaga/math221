\documentclass[main.tex]{subfiles}
\begin{document}
\subsection{Day 36: 11/16/22}

Suppose that $A$ is an invertible $n\times n$ matrix, i.e. $\det A \neq 0$. Then we know that $A\vx = \vec{b}\iff \vx = A^{-1}\vec{b}$, so subbing in $\vec{b} = \vec{e}_j$ for $j = 1, \ldots , n$, we have
\begin{align*}
    A\vx = \vec{e}_1 &\iff \vx = A^{-1}\vec{e}_1 && (\text{column }1\text{ of }A^{-1}) \\
    &\vdots \\
    A\vx = \vec{e}_n &\iff \vx = A^{-1}\vec{e}_n && (\text{column }n\text{ of }A^{-1})
\end{align*}
Concatenating the solution vectors, we can obtain $A^{-1}$.
\begin{example}
    Find the inverse of $A = \begin{pmatrix}
        1 & 4 & 3 \\
        1 & 4 & 5 \\
        2 & 5 & 1
    \end{pmatrix}$.
\end{example}

To obtain this, it suffices to solve:
\begin{enumerate}
    \item $A\vx = \vec{e}_1 \iff \parens{\begin{array}{@{}c | c@{}}
        & 1 \\
        A & 0 \\
        & 0
    \end{array}}\iff \parens{\begin{array}{@{}c | c@{}}
        & d_{11} \\
        \rref A & d_{21} \\
        & d_{31}
    \end{array}}$,
    \item $A\vx = \vec{e}_2 \iff \parens{\begin{array}{@{}c | c@{}}
        & 0 \\
        A & 1 \\
        & 0
    \end{array}}\iff \parens{\begin{array}{@{}c | c@{}}
        & d_{12} \\
        \rref A & d_{22} \\
        & d_{32}
    \end{array}}$,
    \item $A\vx = \vec{e}_2 \iff \parens{\begin{array}{@{}c | c@{}}
        & 0 \\
        A & 0 \\
        & 1
    \end{array}}\iff \parens{\begin{array}{@{}c | c@{}}
        & d_{13} \\
        \rref A & d_{23} \\
        & d_{33}
    \end{array}}$.
    \item Concatenating the solutions, we have
    $A^{-1} = \begin{pmatrix}
        d_{11} & d_{12} & d_{13} \\
        d_{21} & d_{22} & d_{23} \\
        d_{31} & d_{32} & d_{33}
    \end{pmatrix}.$
\end{enumerate}
This can done by a single row reduction of the $3\times 6$ matrix
\begin{align*}
    &\parens{\begin{array}{@{}c c c | c c c@{}}
        1 & 4 & 3 & 1 & 0 & 0 \\
        1 & 4 & 5 & 0 & 1 & 0 \\
        2 & 5 & 1 & 0 & 0 & 1
    \end{array}} \\
    &\iff \parens{\begin{array}{@{}c c c | c c c@{}}
        1 & 4 & 3 & 1 & 0 & 0 \\
        0 & 0 & 2 & -1 & 1 & 0 \\
        0 & -3 & -5 & -2 & 0 & 1
    \end{array}} \iff \parens{\begin{array}{@{}c c c | c c c@{}}
        1 & 4 & 3 & 1 & 0 & 0 \\
        0 & 0 & 1 & -\frac{1}{2} & \frac{1}{2} & 0 \\
        0 & -3 & -5 & -2 & 0 & 1
    \end{array}} \\
    &\iff \parens{\begin{array}{@{}c c c | c c c@{}}
        1 & 4 & 0 & \frac{5}{2} & -\frac{3}{2} & 0 \\
        0 & 0 & 1 & -\frac{1}{2} & \frac{1}{2} & 0 \\
        0 & -3 & 0 & -\frac{9}{2} & \frac{5}{2} & 1
    \end{array}} \iff \parens{\begin{array}{@{}c c c | c c c@{}}
        1 & 4 & 0 & \frac{5}{2} & -\frac{3}{2} & 0 \\
        0 & 0 & 1 & -\frac{1}{2} & \frac{1}{2} & 0 \\
        0 & 1 & 0 & \frac{3}{2} & -\frac{5}{6} & -\frac{1}{3}
    \end{array}} \\
    &\iff \parens{\begin{array}{@{}c c c | c c c@{}}
        1 & 0 & 0 & -\frac{7}{2} & \frac{11}{6} & \frac{4}{3} \\
        0 & 0 & 1 & -\frac{1}{2} & \frac{1}{2} & 0 \\
        0 & 1 & 0 & \frac{3}{2} & -\frac{5}{6} & -\frac{1}{3}
    \end{array}} \iff \parens{\begin{array}{@{}c c c | c c c@{}}
        1 & 0 & 0 & -\frac{7}{2} & \frac{11}{6} & \frac{4}{3} \\
        0 & 1 & 0 & \frac{3}{2} & -\frac{5}{6} & -\frac{1}{3} \\
        0 & 0 & 1 & -\frac{1}{2} & \frac{1}{2} & 0
    \end{array}} \\
    &\implies A^{-1} = \begin{pmatrix}
        -\frac{7}{2} & \frac{11}{6} & \frac{4}{3} \\
        \frac{3}{2} & -\frac{5}{6} & -\frac{1}{3} \\
        -\frac{1}{2} & \frac{1}{2} & 0
    \end{pmatrix}.
\end{align*}

This gives a general approach to solving for a matrix inverse using row reduction:
\begin{enumerate}
    \item Set up the augmented $n\times 2n$ matrix $(A \mid I)$.
    \item Perform a full row reduction to get to $(\rref A \mid D)$.
    \item By last time's theorem, if $\rref A = I$, then $A^{-1} = D$; otherwise, $A$ is not invertible.
\end{enumerate}

\subsubsection{Part I: Eigenvalues and Eigenvectors}

\begin{definition}
    Suppose $A$ is an $n\times n$ matrix. We say $\lambda$ is an \vocab{eigenvalue} of $A$ if it satisfies $\det \parens{A - \lambda I} = 0$.
\end{definition}

\begin{example}
    Is $\lambda = 1$ an eigenvalue of $A = \begin{pmatrix}
        1 & 0 & 2 \\
        0 & 1 & 0 \\
        2 & 0 & 1
    \end{pmatrix}$?
\end{example}

Well, we have
\begin{align*}
    \det \parens{A - \lambda I} &= \det \parens{A - 1I} \\
    &= \det \parens{\begin{pmatrix}
        1 & 0 & 2 \\
        0 & 1 & 0 \\
        2 & 0 & 1
    \end{pmatrix} - \begin{pmatrix}
        1 & 0 & 0 \\
        0 & 1 & 0 \\
        0 & 0 & 1
    \end{pmatrix}} \\
    &= \det\begin{pmatrix}
        0 & 0 & 2 \\
        0 & 0 & 0 \\
        2 & 0 & 0
    \end{pmatrix} = 0
\end{align*}
because there is an empty column, so $\lambda = 1$ is indeed an eigenvalue.

\begin{example}
    What are all the eigenvalues of $A$?
\end{example}

Well, we have
\begin{align*}
    \det \parens{A - \lambda I} &= \det \begin{pmatrix}
        1 - \lambda & 0 & 2 \\
        0 & 1 - \lambda & 0 \\
        2 & 0 & 1 - \lambda
    \end{pmatrix} \\
    &= (-1)^{2 + 2}(1 - \lambda)\det \begin{pmatrix}
        1 - \lambda & 2 \\
        2 & 1 - \lambda
    \end{pmatrix} \\
    &= (1 - \lambda)[(1 - \lambda)^2 - 2^2] \\
    &= (1 - \lambda)(-1 - \lambda)(3 - \lambda).
\end{align*}
Thus, the eigenvalues of $A$ are just $1, -1, 3$.

\begin{lemma}
    Let $A$ be a matrix. $\lambda$ is an eigenvalue of $A$ if and only if there exists a vector $\vec{v} \in \RR^n \setminus\braces{\vec{0}}$ such that $A\vec{v} = \lambda \vec{v}$.
\end{lemma}

\begin{proof}
    We know $\lambda$ is an eigenvalue if and only if $\det (A - \lambda I) = 0 \iff N\det (A - \lambda I) \neq \braces{\vec{0}} \iff \exists\,\vec{v} \neq \vec{0}$ such that $(A - \lambda I)\vec{v} = \vec{0}\implies A\vec{v} - \lambda I \vec{v} \implies A\vec{v} = \lambda I \vec{v} \implies A\vec{v} = \lambda\vec{v}$.
\end{proof}
\end{document}